\documentclass[a4paper, 11pt]{article}

\usepackage{kotex} % Comment this out if you are not using Hangul
\usepackage{fullpage}
\usepackage{hyperref}
\usepackage{amsthm}
\usepackage[capitalise]{cleveref}
\usepackage[numbers,sort&compress]{natbib}

\theoremstyle{definition}
\newtheorem{exercise}{Exercise}

\begin{document}
%%% Header starts
\noindent{\large\textbf{IS-521 Activity Proposal}\hfill
                \textbf{Minjoon Park}} \\
         {\phantom{} \hfill \textbf{dinggul}} \\
         {\phantom{} \hfill Due Date: April 15, 2017} \\
%%% Header ends

\section{Activity Overview}

좋은 엔지니어들의 특징 중 하나는 자신만의 개발 환경을 가지고 있으며
어디서나 구축하여 사용할 수 있다는 점이다. 또한 에디터의 사용 숙련도
또한 높아야 한다. VIM을 에디터로 사용하는 유저가 VIM의 유용한 단축키나
설정을 사용하지 않는 것은 syntax highlight 기능이 있는 윈도우 메모장을
사용하는 것과 크게 다름이 없다. 좋은 도구를 사용하는 것도 중요하지만
좋은 도구를 효율적으로 사용하는 것 역시 중요하다. 따라서 이 Activity
에서는 자신만의 환경 구축과 이 환경을 이식하기 편하게 세팅하는 연습을 하고,
사용하는 에디터의 숙련도를 높이는 것을 목표로 한다.


\section{Exercises}

\begin{exercise} \label{ex:mkset}

  자신이 개발할 때 사용하는 개발환경을 정리한다. 에디터 이외에 자주 사용하는
  프로그램이나(tmux, screen 등) 설정파일등을 꾸미고 적용해본다. 에디터의
  경우에는 다양한 플러그인이 존재하는데 살펴보고 자신의 에디터에 적용해본다.

\end{exercise}

\begin{exercise} \label{ex:dotfiles}

  \cref{ex:mkset} 에서 만든 개발 환경설정들은, 작업하는 머신이 바뀌더라도
  금방 적용하여 사용할 수 있어야 한다. GitHub 에는 많은 사용자들이 이미
  자신의 환경설정을 만들어 공유하고 있다.\cite{dotfiles} 참고하여 직접 만들어
  Activity의 repository에 올려 공유한다.

\end{exercise}

\begin{exercise} \label{ex:adven}

  이미 많은 웹페이지들이 여러가지 종류의 에디터들의 멋지고 효과적인
  단축키에 대해서 잘 설명하고 있다.\cite{usefulVIM} 좋은 정보들이 많지만
  잘 사용하게 되지 않는 이유는 익숙함에서 탈피하는 일이 어렵기 때문이다.
  알고 있는 것만으로 충분히 할 수 있다는 생각과 더 배우는 것이 귀찮다는 생각이
  더 숙련된 에디터 유저가 되는 것의 허들이 된다. 따라서 VIM adventures
  \cite{vimAdventures} 와 같은 방식은 이런 허들을 내리는 목적에 아주 효과적이다.
  (하지만 Emacs 버전은 존재하지 않는 듯 하다) level 이 올라갈수록 더 다양한
  단축키를 배우게 되는데, 특정 레벨까지 클리어 하고 인증하는 것이 \cref{ex:adven}
  의 목표이다.

\end{exercise}

\section{Expected Solutions}

\cref{ex:mkset}, \cref{ex:dotfiles} 의 결과물은, repository 하나와
자신의 개발환경을 설명하는 tex파일 하나가 될 것이다. \cref{ex:adven}
에서는 각 level 별 어떤 키워드를 사용하여 clear를 할 수 있는지에 대한
간략한 설명과 clear 인증 스크린 샷이 하나 포함된 tex 파일 하나가 될 것이다.

\cref{ex:adven}을 제외한 과제들은 개인의 개발환경과 관련된 것이기 때문에
쉽사리 평가하기 어려운 부분이 있다(무엇이 편한지는 주관적이기 때문에).
따라서 자신의 환경설정에 대한 설명이 잘 되어 있는지에 대한 여부와
남들도 쉽게 그 환경설정을 사용할 수 있도록 잘 설계되어 있는지 여부만
(`./install.sh` 스크립트의 존재 유무 등) 확인하는 것이 바람직할 것이다.

\bibliography{references}
\bibliographystyle{plainnat}

\end{document}
